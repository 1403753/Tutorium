%\documentclass{article}
\documentclass{scrartcl}
\usepackage[utf8]{inputenc}
\usepackage{amsmath}
\usepackage{algorithmic}
\usepackage{algorithm}
\usepackage{xcolor}


%%%%%%%%%%%%%%%%%%%%%%%%%%%%%%%%%%%%%%%%%%%%%%%%%%%%%%%%%%%%
%Note Block - source: https://tex.stackexchange.com/questions/94464/note-environment-with-mdframed
%%%%%%%%%%%%%%%%%%%%%%%%%%%%%%%%%%%%%%%%%%%%%%%%%%%%%%%%%%%%
%\usepackage[framemethod=default]{mdframed}

%\global\mdfdefinestyle{exampledefault}{%
%linecolor=lightgray,linewidth=1pt,%
%leftmargin=0cm,rightmargin=0cm,
%}

\newcommand{\fl}{\text{fl}}
\newcommand{\op}{\text{ op }}

\newenvironment{mymdframed}[1]{%
\mdfsetup{%
frametitle={\colorbox{white}{\,#1\,}},
frametitleaboveskip=-\ht\strutbox,
frametitlealignment=\raggedright
}%
\begin{mdframed}[style=exampledefault]
}{\end{mdframed}}

%%%%%%%%%%%%%%%%%%%%%%%%%%%%%%%%%%%%%%%%%%%%%%%%%%%%%%%%%%%%

\title{Submission Guidelines for Homework 2}
\subtitle{VU Numerical High Performance Algorithms, WiSe 2018}
\date{\underline{\emph{due date: 19.11.2018, 18:00}}}

\begin{document}

\maketitle

%        \begin{mymdframed}{Submission Guidelines}
%            \begin{itemize}
%                \item Please submit an anonymized version of your report without your name or your Matrikelnummer appearing in the document. 
%            \end{itemize}
%        \end{mymdframed}

%\section*{Prerequisites}
\begin{enumerate}
	\item Basics:
	\begin{itemize}
		\item Octave users: please use \textit{version 4.4} or higher
		\item Matlab users: please use \textit{version 9.2 (R2017a)} or higher 
		\item Your submission will be evaluated. Please indicate the used environment (Matlab/Octave) and version in your report.
		\item No global variables allowed.
		\item Pay attention to the interface definitions (i.e., use the specified terms. In/output parameters must be in the specified order.)
		\item Your routines should always check the number and types of input arguments.
		\item Do not plot results in predefined routines. Plot results in scripts or self defined routines only.
		\item You can either use your own implementation of the lu-factorization from HW1 or the integrated routines provided by Matlab/Octave.
		\item Measure runtimes of routines only (i.e. do not measure the time needed for memory allocation and initializations). Measurements
have to be done outside the specified routines.
	\end{itemize}

	\item Interface:
	\begin{itemize}
		\item Implement the following interface for the \textit{inverse iteration}:
		\begin{eqnarray*}
			[lambda, v, it, erreval, errres] = invit(n, A, x_0, sigma, eps, maxit, l)
		\end{eqnarray*}
		\item Implement the following interface for the \textit{standard Rayleigh quotient iteration}:
		\begin{eqnarray*}
			[lambda, v, it, erreval, errres] = rqi(n, A, x_0, sigma, eps, maxit, l)				
		\end{eqnarray*}

\item Implement the following interface for the $k^{th}$ iteration variant of \textit{RQI}:
		\begin{eqnarray*}
			[lambda, v, it, erreval, errres] = rqi\_k(n, A, x_0, sigma, eps, maxit, l, k)			
		\end{eqnarray*}	
	
		\begin{itemize}
			\item[*] Description of input parameters: 
			\begin{itemize}
				\item[-] $n$: dimension (scalar)
				\item[-] $A$: $n \times n$ matrix
				\item[-] $x_0$: starting vector of size $n$
				\item[-] $sigma$: shift/eigenvalue approximation (scalar)
				\item[-] $eps$: error tolerance (scalar)
				\item[-] $maxit$: the maximum number of iterations (scalar)
				\item[-] $l$ : reference (true) dominant eigenvalue (scalar)
				\item[-] $k$: a scalar defining the number of $k-1$ iterations before the shift is updated
			\end{itemize}
			\item[*] Description of output parameters: 
			\begin{itemize}
				\item[-] $lambda$: the dominant eigenvalue (scalar)
				\item[-] $v$: the dominant eigenvector of size $n$
				\item[-] $it$: the iteration-number at termination (scalar)
				\item[-] $erreval$: a vector of size $it$ containing the history of relative eigenvalue approximation errors
				\item[-] $errres$: a vector of size $it$ containing the history of relative residuals
			\end{itemize}
		\end{itemize}
		\vspace{0.2cm}
		
		\item Write a script \textit{assignment2.m} to call your routines and plot your results.
	\end{itemize}

	\item Submission:
	\begin{itemize}		
		\item  Upload a single zip archive with all your source code files and your report (as a single
			PDF file named \textit{report.pdf} with all plots and discussions of results) on the course page in Moodle.
		\item Name your archive \texttt{a$<$matriculation\_number$>$.zip} (e.g. \textit{a01234567.zip})
		\item Directories in the archive are not allowed.
		\item A complete submission should include the following files:
		\begin{enumerate}
		\item Routine(s): \textit{invit.m, rqi.m, rqi\_k.m}, self defined routines (optional)
		\item Script(s): \textit{assignment2.m}
		\item Documentation: \textit{report.pdf}
		\end{enumerate}
		
		\end{itemize}
	\end{enumerate}
\end{document}
