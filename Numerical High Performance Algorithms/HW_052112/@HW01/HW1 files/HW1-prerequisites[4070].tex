%\documentclass{article}
\documentclass{scrartcl}
\usepackage[utf8]{inputenc}
\usepackage{amsmath}
\usepackage{algorithmic}
\usepackage{algorithm}
\usepackage{xcolor}


%%%%%%%%%%%%%%%%%%%%%%%%%%%%%%%%%%%%%%%%%%%%%%%%%%%%%%%%%%%%
%Note Block - source: https://tex.stackexchange.com/questions/94464/note-environment-with-mdframed
%%%%%%%%%%%%%%%%%%%%%%%%%%%%%%%%%%%%%%%%%%%%%%%%%%%%%%%%%%%%
%\usepackage[framemethod=default]{mdframed}

%\global\mdfdefinestyle{exampledefault}{%
%linecolor=lightgray,linewidth=1pt,%
%leftmargin=0cm,rightmargin=0cm,
%}

\newcommand{\fl}{\text{fl}}
\newcommand{\op}{\text{ op }}

\newenvironment{mymdframed}[1]{%
\mdfsetup{%
frametitle={\colorbox{white}{\,#1\,}},
frametitleaboveskip=-\ht\strutbox,
frametitlealignment=\raggedright
}%
\begin{mdframed}[style=exampledefault]
}{\end{mdframed}}

%%%%%%%%%%%%%%%%%%%%%%%%%%%%%%%%%%%%%%%%%%%%%%%%%%%%%%%%%%%%

\title{Submission Guidelines for Homework 1}
\subtitle{VU Numerical High Performance Algorithms, WiSe 2018}
\date{\underline{\emph{due date: 22.10.2018, 18:00}}}

\begin{document}

\maketitle

%        \begin{mymdframed}{Submission Guidelines}
%            \begin{itemize}
%                \item Please submit an anonymized version of your report without your name or your Matrikelnummer appearing in the document. 
%            \end{itemize}
%        \end{mymdframed}

%\section*{Prerequisites}
\begin{enumerate}
	\item Basics:
	\begin{itemize}
		\item Octave users: please use \textit{version 4.4} or higher
		\item Matlab users: please use \textit{version 9.2 (R2017a)} or higher 
		\item No global variables allowed.
		\item Pay attention to the interface definitions (i.e., use the specified terms! In/output parameters must be in the specified order!)
		\item Your routines should always check the number and types of input arguments.
		\item Do not plot results in predefined routines! Plot results in scripts or self defined routines only.
	\end{itemize}
	\item Interface:
	\begin{itemize}
		\item For the blocked LU factorization routine implement the following interface:
		\begin{equation*}
			[A, P] = plu(A, n)
		\end{equation*}
		
		\begin{itemize}
			\item Input: $n \times n$ matrix $A$, $n$
			\item Output: 
			\begin{itemize}
				\item $n \times n$ matrices $L$ and $U$ stored in the array $A$
				\item the permutation matrix $P$
			\end{itemize}
		\end{itemize}
		
		\item Additionally, write an evaluation-routine that calls and evaluates your \textit{plu} implementation:
		\begin{equation*}
			[rn, foe, fae, t] = pluStats(A, n)
		\end{equation*}

		\begin{itemize}
			\item Input: $n \times n$ matrix $A$, $n$
			\item Output: 
			\begin{itemize}
				\item the relative residual norm $rn$
				\item the relative forward error $foe$
				\item the relative factorization error $fae$
				\item the runtime $t$
			\end{itemize}
		\end{itemize}
		
		\item Write a script \textit{assignment1.m} to call your routines and plot your results.
	\end{itemize}

	\item Submission:
	\begin{itemize}		
		\item  Upload a single zip archive with all your source code files and your report (as a single
			PDF file named \textit{report.pdf} with all plots and discussions of results) on the course page in Moodle.
		\item Name your archive \texttt{a$<$matriculation number$>$.zip} (e.g. \textit{a01234567.zip})
		\item Directories in the archive are not allowed.
		\item A complete submission should include the following files:
		\begin{enumerate}
		\item Routine(s): \textit{plu.m, pluStats.m}, self defined routines (optional)
		\item Script(s): \textit{assignment1.m}
		\item Documentation: \textit{report.pdf}
		\end{enumerate}
		
		\end{itemize}
	\end{enumerate}
\end{document}
