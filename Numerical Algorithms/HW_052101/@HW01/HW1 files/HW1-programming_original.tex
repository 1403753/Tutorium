%\documentclass{article}
\documentclass{scrartcl}
\usepackage[utf8]{inputenc}
\usepackage{amsmath}
\usepackage{algorithmic}
\usepackage{algorithm}
\usepackage{xcolor}


%%%%%%%%%%%%%%%%%%%%%%%%%%%%%%%%%%%%%%%%%%%%%%%%%%%%%%%%%%%%
%Note Block - source: https://tex.stackexchange.com/questions/94464/note-environment-with-mdframed
%%%%%%%%%%%%%%%%%%%%%%%%%%%%%%%%%%%%%%%%%%%%%%%%%%%%%%%%%%%%
%\usepackage[framemethod=default]{mdframed}

%\global\mdfdefinestyle{exampledefault}{%
%linecolor=lightgray,linewidth=1pt,%
%leftmargin=0cm,rightmargin=0cm,
%}

\newcommand{\fl}{\text{fl}}
\newcommand{\op}{\text{ op }}

\newenvironment{mymdframed}[1]{%
\mdfsetup{%
frametitle={\colorbox{white}{\,#1\,}},
frametitleaboveskip=-\ht\strutbox,
frametitlealignment=\raggedright
}%
\begin{mdframed}[style=exampledefault]
}{\end{mdframed}}

%%%%%%%%%%%%%%%%%%%%%%%%%%%%%%%%%%%%%%%%%%%%%%%%%%%%%%%%%%%%

\title{Homework Sheet 1}
\subtitle{VU Numerical Algorithms, SoSe 2019}
\date{\underline{\emph{due date: 23.4.2019, 18:00}}}

\begin{document}

\maketitle

%        \begin{mymdframed}{Submission Guidelines}
%            \begin{itemize}
%                \item Please submit an anonymized version of your report without your name or your Matrikelnummer appearing in the document. 
%            \end{itemize}
%        \end{mymdframed}


\section*{Programming Exercise}
    The task is to implement an LU factorization-based linear solver in OCTAVE and to evaluate its accuracy for various test matrices. The solver consists of computing the LU decomposition of a square $n \times n$ double precision matrix $A$ such that $A = LU$ with lower triangular $L$ and upper triangular $U$ and subsequent forward and back substitution. In particular:
        
    \subsection*{Part I - LU Decomposition (3 points)}
        First implement the standard "scalar" (unblocked) algorithm (i.e. three nested loops) \textbf{with partial pivoting}.
        
        \begin{algorithm}
            \caption{Pseudo-Code LU Decomposition with partial pivoting}
            \begin{algorithmic}
                \FOR {$k=1$ to $n-1$}
                    \STATE Find index $p$ such that
                    \STATE $|a_{pk}| \geq |a_{ik}| \; for \; k \leq i \leq n$
                    \IF {$p \neq k$}
                        \STATE interchange rows $k$ and $p$
                    \ENDIF
                    \IF {$a_{kk} = 0$}
                        \STATE continue with next $k$
                    \ENDIF
                    \FOR {$i=k+1$ to $n$}
                        \STATE $m_{ik} = a_{ik} / a_{kk}$
                    \ENDFOR
                    \FOR {$j=k+1$ to $n$}
                        \FOR {$i=k+1$ to $n$}
                            \STATE $a_{ij} = a_{ij} - m_{ik}a_{kj}$
                        \ENDFOR
                    \ENDFOR
                \ENDFOR
            \end{algorithmic}
        \end{algorithm}
    
%        \begin{mymdframed}{Note}
            \begin{itemize}
%                \item Also implement partial pivoting (not shown in the pseudo-code).
                \item $U$ is contained in the upper triangle (plus diagonal) of $A$, and the diagonal entries of $L$ are all 1. The subdiagonal entries of $L$ are given by the scalars $m_{ik}$ (i.e. $L(i,k) = m_{ik}$). For storage efficiency, we can store $L$ in the lower triangle of $A$, and thus $A(i,k)$ has to be overwritten with $m_{ik}$.
            \end{itemize}
%        \end{mymdframed}
    
    \subsubsection*{Detailed remarks:}
    \begin{enumerate}
        \item \textit{Accuracy}: Verify the correctness of your $LU$ factorization by evaluating the relative residual
        \begin{align*}
            R = \frac{\|P^TLU - A\|_{1}}{\|A\|_{1}}
        \end{align*}
        where $\|\cdot\|_{1}$ is the maximum absolute column sum of a matrix:
        \begin{align*}
            \|M\|_{1} = \underset{j = 1, \dots, n}{max} \; \sum_{i=1}^{n} |M_{ij}|
        \end{align*}
        
        Plot these residuals $R$ for all problem sizes you experimented with.
        \emph{When plotting residuals, always use a \textbf{logarithmic scale} along the \textbf{y-axis}!}
        
        \item Use randomly generated matrices $A$ as input, but please specify clearly in your report how you generated your test matrices!
        
        \item Produce separate routines (i.\,e., preparation of the input matrices, $LU$ decomposition). Write modular code! Use consistent interfaces for your routines:
        \begin{enumerate}
            \item Input: $A, n$
            \item Output: $n \times n$ matrices $L$ and $U$ stored in the array $A$ ($A=LU$) and the permutation matrix $P$.
        \end{enumerate}
        
    %    \item \textit{Report}: Summarize your work in a written report, summarizing your measurements and an interpretation of your results.
        
        \item What is NOT allowed:
            \begin{enumerate}
                \item Do not use any existing code which you did not write yourself!
                \item Do not try to exploit any special structure in the input data. The code has to be generic and should run for all possible input matrices $A$.
            \end{enumerate}
    \end{enumerate}
    
    \subsection*{Part II - Solving a Triangular Linear Systems (1 point)}
        \begin{enumerate}
            \item \textbf{Forward substitution}: Write a routine which solves a given $n \times n$ lower triangular linear system $Lx = b$ for $x$.
            \item \textbf{Back substitution}: Write another routine which solves a given $n \times n$ upper triangular linear system $Ux = b$ for $x$.
            \item Evaluate the accuracy of your codes for increasing $n$ in terms of the relative residual and the relative forward error. \emph{(For the definition of relative residual and relative forward error please see Part III!)}
            
            For these experimental evaluations, use randomly generated (non-singular) $L$ and $U$ and determine $b$ such that the exact solution $x$ is a vector of all ones: $x = (1, 1, \dots, 1, 1)^{T}$.  
        \end{enumerate}

    \subsection*{Part III - Numerical Accuracy of LU-Based Linear Solver (4 points)}
    The main purpose of this part is to experimentally evaluate the numerical accuracy of the linear systems solver you implemented in Parts I and II for different test matrices and to compare it with the built-in solver from OCTAVE. 
    You can solve a linear system $Ax=b$ for $x$ using the $\setminus$ operator (e.g. $x = A \setminus b$).
 
    \begin{enumerate}
        \item Take your LU factorization from Part I and combine it with your triangular linear systems solvers from Part II in order to get a complete LU-based linear solver.
        \item Input data for your experiments:
        \begin{enumerate}
            \item Generate random test matrices $S$ with entries uniformly distributed in the interval [-1,1].
            \item Generate test matrices $H$ which are defined by
            \begin{align*}
                H_{ij} := \frac{1}{i+j-1} \quad for \;i=1,\dots,n \; \mbox{and} \; j=1, \dots, n.
            \end{align*}
            \item In all your test cases, determine the corresponding right hand side $b$ of length $n$ such that the exact solution $x$ of the linear system is a vector of all ones: $x = (1, 1, \dots, 1, 1)^{T}$.
        \end{enumerate}
        \item Solve the linear systems $Sx = b$ and $Hx=b$ with your LU-based linear solver and the built-in OCTAVE solver and evaluate the numerical accuracy of the computed solution.
        \begin{enumerate}
            \item \textit{Problem sizes}: Start with $n=2,3,4,5,\dots,10$ then incease in increments of $5$. For $n > 50$ you can further increase the increment. The largest value of $n$ should be as large as possible (so that your code terminates within a reasonable time).
            \item \textit{Accuracy}: For the computed solution $\hat{x}$, evaluate the relative residual $r$:
            \begin{align*}
                r := \frac{||M\hat{x} - b||_1}{||b||_1}
            \end{align*}
            ($M$ is $S$ or $H$)
            as well as the relative forward error $f$:
            \begin{align*}
                f := \frac{||\hat{x} - x||_1}{||x||_1}. 
            \end{align*}
        \end{enumerate}
        \item For both your and the OCTAVE solver generate the following plots for the different test matrices:
            \begin{enumerate}
            \item Relative residual and relative forward error in $\hat{x}$ vs. $n$: One figure for both accuracy metrics for matrix type $S$, another figure for both accuracy metrics for matrix type $H$.
            \end{enumerate}
        \item Interpret and explain your experimental results in your report. Do you think that there is a fundamental difference in the numerical accuracy which your LU-based linear solver achieves for the two types of test matrices? If yes, explain the reasons for this difference. How does your solver compare to the OCTAVE version?
    \end{enumerate}
\end{document}
