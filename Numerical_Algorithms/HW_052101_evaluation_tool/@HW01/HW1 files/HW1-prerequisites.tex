%\documentclass{article}
\documentclass{scrartcl}
\usepackage[utf8]{inputenc}
\usepackage[bottom]{footmisc}
\usepackage{amsmath}
\usepackage{algorithmic}
\usepackage{algorithm}
\usepackage{xcolor}
\usepackage[colorlinks = true,
            linkcolor = blue,
            urlcolor  = blue,
            citecolor = blue,
            anchorcolor = blue]{hyperref}

%%%%%%%%%%%%%%%%%%%%%%%%%%%%%%%%%%%%%%%%%%%%%%%%%%%%%%%%%%%%
%Note Block - source: https://tex.stackexchange.com/questions/94464/note-environment-with-mdframed
%%%%%%%%%%%%%%%%%%%%%%%%%%%%%%%%%%%%%%%%%%%%%%%%%%%%%%%%%%%%
%\usepackage[framemethod=default]{mdframed}

%\global\mdfdefinestyle{exampledefault}{%
%linecolor=lightgray,linewidth=1pt,%
%leftmargin=0cm,rightmargin=0cm,
%}

\newcommand{\fl}{\text{fl}}
\newcommand{\op}{\text{ op }}

\newenvironment{mymdframed}[1]{%
\mdfsetup{%
frametitle={\colorbox{white}{\,#1\,}},
frametitleaboveskip=-\ht\strutbox,
frametitlealignment=\raggedright
}%
\begin{mdframed}[style=exampledefault]
}{\end{mdframed}}

%%%%%%%%%%%%%%%%%%%%%%%%%%%%%%%%%%%%%%%%%%%%%%%%%%%%%%%%%%%%

\title{Submission Guidelines for Homework 1}
\subtitle{VU Numerical Algorithms, SoSe 2019}
\date{\underline{\emph{due date: 23.4.2019, 18:00}}}

\begin{document}
\maketitle

%        \begin{mymdframed}{Submission Guidelines}
%            \begin{itemize}
%                \item Please submit an anonymized version of your report without your name or your Matrikelnummer appearing in the document.
%            \end{itemize}
%        \end{mymdframed}
\section*{Prerequisites}
\begin{enumerate}
	\item Basics:
	\begin{itemize}
		\item Please use Octave\footnote{Octave download page: \url{https://www.gnu.org/software/octave/download.html}} \textit{version 4.4} or higher and indicate the Octave version in your report. Your submission will be evaluated.
		\item Do not import additional packages and do not use global variables.
		\item Pay attention to the interface definitions, i.e., use the specified terms. In/output parameters must be in the specified order.
		\item Your routines should always check the number and types of input arguments.
		\item Do not plot results in predefined routines! Plot results in scripts or self defined routines only.
		\item Do not exploit any special structure in the input data. Your routines must be generic and have to work for all $n > 1$.
		\item Do not use any existing code which you did not write yourself!
		\item You can define your own routines in order to write modular code but please stay consistent with the predefined interface.
	\end{itemize}
	
	\item Interface:
	\begin{itemize}
		\item Mandatory for \textbf{all Parts}:
			\begin{enumerate}
        		\item Write a script \textit{assignment1.m} to test your routines and plot your results.\\
				\item Create a file $accuracy.m$ of the following form:
     			\begin{align*}
            		\texttt{[z] = accuracy(X, Y)}
        		\end{align*}
				\begin{itemize}
            		\item[-] Input: $X$ and $Y$ are either both $n \times n$ matrices or both vectors of size $n$.
            		\item[-] Output: scalar $z$, with
            		\begin{equation*}
	         			z := \frac{||X - Y||_1}{||Y||_1}.
            		\end{equation*}
        		\end{itemize}
        		\textbf{Remark}: Use this routine to verify the correctness of your $LU$ factorization in \textbf{Part I} and to compute the \textit{relative residual} and \textit{relative forward error} in \textbf{Part II} resp. \textbf{Part III}.\\
		\end{enumerate}
	
		\item Mandatory for \textbf{Part I}:
			\begin{enumerate}
				\item Create a file $plu.m$ for the $LU$ factorization with partial pivoting:
        		\begin{align*}        	 
        			\texttt{[A, P] = plu(A, n)}
				\end{align*}
        		\begin{itemize}
            		\item[-] Input: $n \times n$ matrix $A, n$
            		\item[-] Output: $n \times n$ matrices $L$ and $U$ stored in the array $A$ ($A=P^TLU$) and the permutation matrix $P$.\\
        		\end{itemize}
        	\end{enumerate}

		\item Mandatory for \textbf{Part II}:
			\begin{enumerate}
				\item Create two files \textit{solveL.m} / \textit{solveU.m} for solving a lower / upper triangular system:
            	\begin{align*}
            		\texttt{[x] = solveL(B, b, n)}\\
					\texttt{[x] = solveU(B, b, n)}
				\end{align*}
        		\begin{itemize}
            		\item[-] Input: $n \times n$ matrix $B$, the right hand side vector $b$ of size $n$, $n$
            		\item[-] Output: the solution vector $x$.
        		\end{itemize}
		\textbf{Remark}: The input matrix $B$ has special structure $B = L + U - I$, where $L$ (with fixed ones in the diagonal) and $U$ are lower resp. upper triangular matrices and $I$ is the Identity.\\
			\end{enumerate}

		\item Mandatory for \textbf{Part III}:
		 \begin{enumerate}
		 	\item Create a file \textit{linSolve.m} for solving a linear system:
		 	\begin{align*}
				\texttt{[x] = linSolve(A, b, n)}
			\end{align*}
			\begin{itemize}
				\item[-] Input: $n \times n$ nonsingular matrix $A$, the right hand side vector $b$ of size $n$, $n$.
            	\item[-] Output: the solution vector $x$.        	
        	\end{itemize}
\textbf{Remark}: This routine must incorporate $\textbf{plu.m}$, $\textbf{solveL.m}$ and $\textbf{solveU.m}$ from previous Parts!\\
		\end{enumerate}
	\end{itemize}

	\item Submission:	
	\begin{itemize}		
		\item  Upload a single zip archive with all your source code files and your report (as a single
			PDF file named \textit{report.pdf} with all plots and discussions of results) on the course page in Moodle.
		\item Name your archive \texttt{a$<$matriculation number$>$\_$<$last name$>$.zip}\\
		(e.g. \textit{a01234567\_mustermann.zip})
		\item Directories in the archive are not allowed.
		\item A complete submission should include the following files:
		\begin{enumerate}
		\item Routines: \textit{accuracy.m, linSolve.m, plu.m, solveL.m, solveU.m}, self defined routines (optional)
		\item Script: \textit{assignment1.m}
		\item Documentation: \textit{report.pdf}
		\end{enumerate}
		
		\end{itemize}
	\end{enumerate}

\section*{Octave cheat sheet}
Below is a list of recommendations that might help with your implementation:
\begin{itemize}

\item You should at least be aware of the following routines provided by Octave:\\

$validateattributes,\,nargin,\,norm,\,tril,\,triu,\,semilogy,\,\ldots$\\

All Octave routines are described here:\\
\url{https://octave.sourceforge.io/list_functions.php}

\item We recommend vectorization to simplify and optimize your code (i.e., eliminate for loops whenever possible):\vspace{.3cm}\\
\url{https://octave.org/doc/v4.0.1/Basic-Vectorization.html}

\item For efficiency, the LU factorization algorithm usually operates on a permutation vector (instead of a whole matrix) during the factorization process and creates the permutation matrix at the very end. You might consider using a permutation vector as well (although, this is not mandatory). Creating permutation matrices is explained here:\vspace{.3cm}\\
\url{https://octave.org/doc/v4.4.1/Creating-Permutation-Matrices.html}


\end{itemize}
\end{document}
